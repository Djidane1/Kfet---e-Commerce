\documentclass[twoside,UTF8]{EPURapport}
%\usepackage{listings}

%\renewcommand{\lstlistlistingname}{Liste des codes}
%\renewcommand{\lstlistingname}{Code}

%\addextratables{%
%	\lstlistoflistings
%}

%\swapAuthorsAndSupervisors
\usepackage{float}


\thedocument{Rapport de projet}{Développement d'un site de vente et de gestion pour la Kfet de Polytech}{Site web Kfet}

\grade{Département Informatique\\ 5\ieme{} année\\ 2011 - 2012}

\authors{%
	\category{Étudiants}{%
		\name{Loïc CARNEY} \mail{loic.carney@etu.univ-tours.fr}
		\name{Sébastien LACROIX} \mail{sebastien.lacroix@etu.univ-tours.fr}
	}
	\details{DI5 2008 - 2009}
}

\supervisors{%
	\category{Encadrants}{%
		\name{Alexandre LISSY} \mail{alexandre.lissy@univ-tours.fr}
	}
	\details{Université François-Rabelais, Tours}
}

\abstracts{Description en français}
{Mots clés français}
{Description en anglais}
{Mots clés en anglais}

\begin{document}

\chapter{Introduction}

    \paragraph{}Dans le cadre de notre dernière année d'étude à Polytech'Tours, nous devions réaliser un projet lié aux technologies web. Nous avons décider de proposer notre propre sujet. L'un de nous ayant été trésorier à la Kfet du département Informatique de l'école, nous étions au courant que toute la gestion de la Kfet se fait sur papier et manuellement. Aucune gestion automatique n'est en place et aucune gestion des stocks n'est effectuée pour le moment. Nous avons donc proposer comme sujet : la réalisation d'un site web permettant de répondre à tous les besoins de gestion de la Kfet: en commençant par la vente jusqu'à la gestion des stocks et en passant par la gestion des dettes.

    \paragraph{}Pour réaliser le projet web, nous avons décidé d'utiliser une technologie web que nous ne connaissions pas : le  framework web Django qui est basé sur le langage interprêté Python. Le but de Django est de donner au développeur web des outils simples et rapides, en effet son slogan est "Le framework web pour les perfectionnistes sous pression".

    \paragraph{}La Kfet de Polytech est ouverte à toutes les pauses. Lors des pauses d'inter-cours elle vend des en-cas, confiseries et des boissons chaudes,et lors de la pause de midi elle sert des plats chauds. Les stocks sont donc assez conséquents et varient fortement, la problématique de la gestion des stocks est un des éléments qui nous a incité à réaliser ce projet.

    \paragraph{}Nous allons donc vous présenter tout d'abord le projet au travers du cahier des charges puis nous détaillerons le fonctionnement de chaque module mis en place.

\chapter{Cahier des charges}

\chapter{Module Commandes}

\chapter{Module Gestion Stocks}

\chapter{Module Ventes}

\chapter{Module Administration}

\chapter{Module dettes}

\chapter{Conclusion}

\annexes

\end{document}

