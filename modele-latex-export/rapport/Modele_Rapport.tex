\documentclass[twoside,UTF8]{EPURapport}
%\usepackage{listings}

%\renewcommand{\lstlistlistingname}{Liste des codes}
%\renewcommand{\lstlistingname}{Code}

%\addextratables{%
%	\lstlistoflistings
%}

%\swapAuthorsAndSupervisors
\usepackage{float}


\thedocument{Rapport de projet}{Développement d'un site de vente et de gestion pour la Kfet de Polytech}{Site web Kfet}

\grade{Département Informatique\\ 5\ieme{} année\\ 2011 - 2012}

\authors{%
	\category{Étudiants}{%
		\name{Loïc CARNEY} \mail{loic.carney@etu.univ-tours.fr}
		\name{Sébastien LACROIX} \mail{sebastien.lacroix@etu.univ-tours.fr}
	}
	\details{DI5 2008 - 2009}
}

\supervisors{%
	\category{Encadrants}{%
		\name{Alexandre LISSY} \mail{alexandre.lissy@univ-tours.fr}
	}
	\details{Université François-Rabelais, Tours}
}

\abstracts{Description en français}
{Mots clés français}
{Description en anglais}
{Mots clés en anglais}

\begin{document}

\chapter{Introduction}

    \paragraph{}Lors de notre dernière année d'étude à Polytech'Tours nous devions réaliser un projet lié au technologie web. Nous avons de plus décider de proposer notre propre sujet. L'un de nous ayant été trésorier à la Kfet nous étions au courant que toute la gestion se fait sur papier et manuellement. Aucune gestion automatique n'est en place et aucune gestion des stocks n'est effectuée pour le moment. Nous avons donc proposer comme sujet la réalisation d'un site web permettant de répondre à tous les besoins de gestion de la Kfet: de la vente à la gestion des stocks en passant par les dettes.

    \paragraph{}Afin de correspondre au projet web nous avons décider d'utiliser une technologie web que nous en connaissions pas : le Python à travers un framework nommé Django. Le but de Django est le développement web simple et rapide, en effet son slogan est "Le framework web pour les perfectionnistes sous pression".

    \paragraph{}La Kfet de Polytech est ouverte à la toutes les pauses. Lors des pauses d'inter-cours elle vend des en-cas, confiseries et des boissons chaudes,et lors de la pause de midi elle sert des plats chauds. Les stocks sont donc assez conséquents et varie fortement, la problématique de la gestion des stocks est le but principal de ce projet.

    \paragraph{}Nous allons donc vous présenter notre projet tout d'abord au travers du cahier des charges puis nous détaillerons le fonctionnement de chaque module mis en place.

\chapter{Cahier des charges}

\chapter{Module Commandes}

\chapter{Module Gestion Stocks}

\chapter{Module Ventes}

\chapter{Module Administration}

\chapter{Module dettes}

\chapter{Conclusion}

\annexes

\end{document}

